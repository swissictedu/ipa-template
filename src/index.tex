% !TeX root = index.tex
% define document class
\documentclass{scrreprt}

% set variables
\newcommand{\varAuthor}{Max Muster}
\newcommand{\varCandidate}{\textbf{\varAuthor} \\ max.muster@example.com} % Kandidat/in
\newcommand{\varResponsibleSpecialist}{\textbf{Franziska Müller} \\ franziska.mueller@example.com} % verantwortliche Fachkraft
\newcommand{\varVocationalTrainer}{\textbf{Thea Meier} \\ thea.meier@example.com} % Berufsbildner/in
\newcommand{\varPrimaryExpert}{\textbf{Franz Krebs} \\ franz.krebs@example.com} % Hauptexperte
\newcommand{\varSecondaryExpert}{\textbf{Adrian Kuster} \\ adrian.kuster@example.com} % Nebenexperte
\newcommand{\varCompany}{Beispiel AG} % Firmenname
\newcommand{\varCompanyDepartment}{Entwicklung} % Abteilungsname
\newcommand{\varTitle}{Meine IPA}
\newcommand{\varVersion}{0.1} % Versionsnummer: Pro IPA-Tag um 0.1 erhöhen
\newcommand{\varExaminationBoard}{Prüfungskomission 19} % Prüfungsorganisation
\newcommand{\varExaminationBoardDepartment}{Informatik Applikationsentwicklung} % Fachrichtung

% apply ipa package
\usepackage{../lib/ipa}
\usepackage{../lib/tikz-uml}

\lohead[\varTitle]{\varTitle}
\lofoot[\today]{\today}
% see B6.5
\cfoot[\varAuthor\ / \varCompany\\Version \varVersion]{\varAuthor\ / \varCompany\\Version \varVersion}
\rofoot[Seite \pagemark{} von \pageref{LastPage}]{Seite \pagemark{} von \pageref{LastPage}}

% load sources
\addbibresource{sources.bib}

% make glossaries
\makenoidxglossaries

% define glossary entries
% see B6.6
\newglossaryentry{Organigramm}{
  name={Organigramm},
  description={Ein Organigramm stellt eine Organisation und deren Aufbauorganisation grafisch dar.}
}

% create document
\begin{document}

  % set page numbering
  \pagenumbering{roman}

  % set page style
  \pagestyle{scrheadings}

  % include title page
  \thispagestyle{empty}
  
  \begin{center}
  \makebox[\textwidth]{\includegraphics[width=1.1\paperwidth]{images/title.jpg}}
\end{center}
\vspace*{1cm}
{\fontsize{28}{28}\selectfont \varTitle}\\[0.25cm]
{\fontsize{16}{16}\selectfont \varAuthor}\\[0.25cm]
{\fontsize{16}{16}\selectfont \varCompany}

  \newpage
  \TileWallPaper{\paperwidth}{\paperheight}{images/background.pdf}

  % see A1.3
  % see B6.3
  % generate the table of contents
  \tableofcontents

  % finish page
  \clearpage

  % use the arabic numbering system
  \pagenumbering{arabic}

  % reset page counter
  \setcounter{page}{1}

  % see B6.1a
  % create a phantom toc entry for "Umfeld und Ablauf"
  \clearpage\phantomsection\addcontentsline{toc}{part}{Umfeld und Ablauf}

  % [2]: Seite 11
\chapter{Aufgabenstellung}

Dieses Kapitel beinhaltet die komplette Aufgabenstellung im originalen Wortlaut.

\section{Ausgangslage}

\lipsum[1-2]

\section{Detaillierte Aufgabenstellung}

\lipsum[3-9]

\section{Mittel und Methoden}
\lipsum[10]
  % [2]: Seite 11
\chapter{Deklaration}

Folgender Abschnitt beschreibt die Vorkenntnisse des Kandidaten und dessen Vorbereitung.

\section{Vorkenntnisse}

\lipsum[11]

\section{Vorarbeiten}

\lipsum[12]

\section{Neue Lerninhalte}

\lipsum[13]

\section{Arbeiten in den letzten 6 Monaten}

\lipsum[14]

  % [1]: Kriterium B6
\chapter{Projektaufbauorganisation}

  % see A1.2
% see A3
% see B6.2b
\begin{landscape}
  \chapter{Zeitplan}
  Der Zeitplan basiert auf der Vorlage \cite{Buhler_ipa-timetable_2022} und zeigt mit einer Auflösung von zwei Stundenblöcken die geplanten und getätigten Aufwände.
  \begin{figure}[H]
    \begin{center}
      \includegraphics[width=1.55\textheight]{../res/timeplan.pdf}
    \end{center}
    \caption[\enquote{Zeitplan} erstellt mit Google Sheets]{Zeitplan}
    \label{fig:timeplan}
  \end{figure}
\end{landscape}

  % see A2.1
% see B6.2c
\chapter{Arbeitsjournal}

\section{Tag 1}
\begin{tabularx}{\textwidth}[H]{|c|X|}
  \hline
  Erledigte Arbeiten & \lipsum[23] \\ \hline
  Ungeplante Arbeiten & \lipsum[24] \\ \hline
  Erfolge & \lipsum[25] \\ \hline
  Misserfolge & \lipsum[26] \\ \hline
  Hilfestellungen & \lipsum[27] \\
  \hline
\end{tabularx}

\newpage

\section{Tag 2}
\begin{tabularx}{\textwidth}[H]{|c|X|}
  \hline
  Erledigte Arbeiten & \lipsum[23] \\ \hline
  Ungeplante Arbeiten & \lipsum[24] \\ \hline
  Erfolge & \lipsum[25] \\ \hline
  Misserfolge & \lipsum[26] \\ \hline
  Hilfestellungen & \lipsum[27] \\
  \hline
\end{tabularx}

\newpage

\section{Tag 3}
\begin{tabularx}{\textwidth}[H]{|c|X|}
  \hline
  Erledigte Arbeiten & \lipsum[23] \\ \hline
  Ungeplante Arbeiten & \lipsum[24] \\ \hline
  Erfolge & \lipsum[25] \\ \hline
  Misserfolge & \lipsum[26] \\ \hline
  Hilfestellungen & \lipsum[27] \\
  \hline
\end{tabularx}

\newpage

\section{Tag 4}
\begin{tabularx}{\textwidth}[H]{|c|X|}
  \hline
  Erledigte Arbeiten & \lipsum[23] \\ \hline
  Ungeplante Arbeiten & \lipsum[24] \\ \hline
  Erfolge & \lipsum[25] \\ \hline
  Misserfolge & \lipsum[26] \\ \hline
  Hilfestellungen & \lipsum[27] \\
  \hline
\end{tabularx}

\newpage

\section{Tag 5}
\begin{tabularx}{\textwidth}[H]{|c|X|}
  \hline
  Erledigte Arbeiten & \lipsum[23] \\ \hline
  Ungeplante Arbeiten & \lipsum[24] \\ \hline
  Erfolge & \lipsum[25] \\ \hline
  Misserfolge & \lipsum[26] \\ \hline
  Hilfestellungen & \lipsum[27] \\
  \hline
\end{tabularx}

\newpage

\section{Tag 6}
\begin{tabularx}{\textwidth}[H]{|c|X|}
  \hline
  Erledigte Arbeiten & \lipsum[23] \\ \hline
  Ungeplante Arbeiten & \lipsum[24] \\ \hline
  Erfolge & \lipsum[25] \\ \hline
  Misserfolge & \lipsum[26] \\ \hline
  Hilfestellungen & \lipsum[27] \\
  \hline
\end{tabularx}

\newpage

\section{Tag 7}
\begin{tabularx}{\textwidth}[H]{|c|X|}
  \hline
  Erledigte Arbeiten & \lipsum[23] \\ \hline
  Ungeplante Arbeiten & \lipsum[24] \\ \hline
  Erfolge & \lipsum[25] \\ \hline
  Misserfolge & \lipsum[26] \\ \hline
  Hilfestellungen & \lipsum[27] \\
  \hline
\end{tabularx}

\newpage

\section{Tag 8}
\begin{tabularx}{\textwidth}[H]{|c|X|}
  \hline
  Erledigte Arbeiten & \lipsum[23] \\ \hline
  Ungeplante Arbeiten & \lipsum[24] \\ \hline
  Erfolge & \lipsum[25] \\ \hline
  Misserfolge & \lipsum[26] \\ \hline
  Hilfestellungen & \lipsum[27] \\
  \hline
\end{tabularx}

\newpage

\section{Tag 9}
\begin{tabularx}{\textwidth}[H]{|c|X|}
  \hline
  Erledigte Arbeiten & \lipsum[23] \\ \hline
  Ungeplante Arbeiten & \lipsum[24] \\ \hline
  Erfolge & \lipsum[25] \\ \hline
  Misserfolge & \lipsum[26] \\ \hline
  Hilfestellungen & \lipsum[27] \\
  \hline
\end{tabularx}

\newpage

\section{Tag 10}
\begin{tabularx}{\textwidth}[H]{|c|X|}
  \hline
  Erledigte Arbeiten & \lipsum[23] \\ \hline
  Ungeplante Arbeiten & \lipsum[24] \\ \hline
  Erfolge & \lipsum[25] \\ \hline
  Misserfolge & \lipsum[26] \\ \hline
  Hilfestellungen & \lipsum[27] \\
  \hline
\end{tabularx}

  % see B6.1a
  % create a phantom toc entry for "Projekt"
  \clearpage\phantomsection\addcontentsline{toc}{part}{Projekt}

  % See B1
\chapter{Kurzfassung}

Die Kurzfassung gibt einen Überblick über das vorliegende Projekt.

\section{Ausgangssituation}

\lipsum[20]

\section{Umsetzung}

\lipsum[21-22]

\section{Ergebnis}

\lipsum[22]
  % see A1.1a
\chapter{Informieren}

Die nachfolgende Dokumentation baut auf der Vorlage \cite{Buhler_ipa-template_2022} auf.
  % see A1.4
\chapter{Planen}
  \input{chapters/decide}
  \chapter{Realisieren}

Verschiedene vorkonfigurierte Pakete helfen den Bericht, speziell die Realisierung, ansprechend zu formatieren und gestalten:

Aufruf von einer Aktion über ein Menü:
\menu{Extras > Settings > Rulers} \\
Drücken von Tastenkombinationen:
\keys{CTRL + R} \\
Verzeichnispfade:
\directory{C:/Windows/system32/hosts.txt} \\
Quellcode:

\begin{codebox}[]
  \begin{minted}{javascript}
// Sortieren eines Arrays mit BubbleSort
let bubbleSort = (inputArr) => {
    let len = inputArr.length;
    for (let i = 0; i < len; i++) {
        for (let j = 0; j < len; j++) {
            if (inputArr[j] > inputArr[j + 1]) {
                let tmp = inputArr[j];
                inputArr[j] = inputArr[j + 1];
                inputArr[j + 1] = tmp;
            }
        }
    }
    return inputArr;
};
  \end{minted}
\end{codebox}
  \input{chapters/check}
  \input{chapters/evaluate}

  % see B6.6
  % create a phantom toc entry for the index/glossary table
  \clearpage\phantomsection\addcontentsline{toc}{part}{Glossar}

  % generate glossary
  \printnoidxglossary[title={Glossar}]

  % create a phantom toc entry for the figures table
  \clearpage\phantomsection\addcontentsline{toc}{part}{Abbildungsverzeichnis}

  % generate figures table
  \listoffigures

  % create a phantom toc entry for the literature table
  \clearpage\phantomsection\addcontentsline{toc}{part}{Literaturverzeichnis}

  % generate bibliography
  \printbibliography[title=Literaturverzeichnis]

  % defines the beginning of the appendix
  \appendix

  % create a phantom toc entry for "Projekt"
  \clearpage\phantomsection\addcontentsline{toc}{part}{Anhang}

  % see B6.1b
\chapter{Quellcode}

\end{document}
